% 
%  chapter1.tex
%  
%  Created by Matilde Pós-de-Mina Pato on 2012/10/09.
%
\chapter{Introdução}
\label{cha:introduction}

\section{Linguagem e estilo}
Qualquer trabalho deverá ser escrito como um artigo, i.e. a linguagem deve ser clara, objectiva, escrita em discurso directo e com frases curtas \cite{gustavii2016write, glasman2010science}.

A redação deve ser feita com frases curtas e objectivas, organizadas de acordo com a estrutura do trabalho, dando destaque a cada uma das partes abordadas, assim apresentadas: Introdução - Informar, em poucas palavras, o contexto em que o trabalho se insere, sintetizando a problemática estudada. Objetivo - Deve ser explicitado claramente. Métodos - Destacar os procedimentos metodológicos adoptados. Resultados - Destacar os mais relevantes para os objectivos pretendidos. Os trabalhos de natureza quantitativa devem apresentar resultados numéricos, assim como seu significado estatístico. Conclusões - Destacar as conclusões mais relevantes, os estudos adicionais recomendados e os pontos positivos e negativos que poderão influir no conhecimento. 

\paragraph{Parágrafo}
\begin{enumerate}
\item Um parágrafo  deve iniciar-se com uma frase curta e que contém a informação principal. As restantes devem acompanhar o conteúdo apresentado na primeira. A última deve fazer a ligação ao parágrafo seguinte.
\item Os parágrafos devem interligar-se de forma lógica.
\end{enumerate}

\paragraph{Estrutura}

O limite de páginas para cada UC será estipulado pelo docente e pode contemplar as seguintes partes: 
\begin{enumerate}
\item[Título] curto mas não genérico.
\item[Capa] Use a capa apresentada neste documento. Complete a informação apresentada no ficheiro \texttt{datas.tex}.
\item[Resumo] Faça um resumo dos conteúdos do trabalho e apresente as conclusões básicas, \texttt{resumo.tex} e \texttt{abstract.tex} (se necessário, c.c. comente a linha 155 de \texttt{template.tex}).
\item[Índice] Indique as páginas dos títulos e subtítulos, figuras ou tabelas. O código que suporta algum parágrafo deve constar no índice respectivo, de Listagens.
\item[Introdução] Contextualize o tema e indique o objectivo de estudo.
\item[Desenvolvimento] Descreva as definições, modelos e teorias suportados por referências bibliográficas. 
\item[Conclusão] Sintetize os aspectos relevantes.
\item[Bibliografia] Escreva todas as referências indicadas no texto.
\item[Anexos] Use os anexos para colocar outras informações que considere oportunas, mas não relevantes o suficiente para colocar no corpo do documento. 
\end{enumerate}

\paragraph{Bibliografia}
As referências são listadas pela ordem alfabética do apelido dos autores e depois por ordem cronológica quando o nome se repetir, \textit{package: plainnat}. 

\begin{enumerate}
\item[Livros] Deve constar o nome original de um livro escrito em língua estrangeira. Pede-se que seja inserido as páginas consultadas. 
\item[websites] Os sítios da internet consultados também devem constar nas referências. Pede-se que seja introduzido o dia de consulta do mesmo.
\item[Artigos] Deve citar aqueles que se encontram indexados e submetidos a revisão independente. No instituto Thomson Reuters são fornecidas listas de toda a bibliografia que obedece a esse grau de exigência.
\item[Trabalhos] Se a citação for relativa a uma comunicação pessoal, então faça-o do modo seguinte: M. Mjhdsh (comunicação pessoal, 13 de Março 2017). Noutros trabalhos deve constar a seguinte informação: ``Dissertação (ou, Relatório) de Mestrado (ou Doutoramento, ou Final de Curso) não publicada(o). 
\end{enumerate}

Por uma questão de simplificação, pode recorrer ao  Google Académico\footnote{scholar.google.pt/} e retirar a informação que consta no sítio \texttt{Citar} no formato BibTeX e inserir no ficheiro \texttt{bibliography.bib}. Por exemplo:
\footnotesize
\begin{verbatim}
@book{gustavii2016write,
  title={How to write and illustrate a scientific paper},
  author={Gustavii, Bj{\"o}rn},
  year={2016},
  publisher={Cambridge University Press}
}
\end{verbatim}

\paragraph{Relatórios de Sistemas de Informação}

Caracteres especiais para Álgebra Relacional

\begin{table}[!htb]
\begin{minipage}{.5\textwidth}
 \caption{Simbolos lógicos e outros}
 \centering  
\begin{tabular}{|l|c|r|}
\hline
Logical AND & 	$\wedge$ \\
Logical OR & 	$\vee$ \\
Logical NOT & 	$\neg$ \\
null & 			$\omega$ \\
\hline
\end{tabular}
\vskip1em
\caption{Operadores unários}
 \centering  
\begin{tabular}{|l|c|r|}
\hline
selecção &  $\select_{cname<cname2} E$ \\
projecção &  $\project_{cname} E$ \\
função de agregação &  $_{g_1,g_2,\ldots} \aggregatefn_{h_1,h_2,\ldots,h_m}$ \\
\hline
\end{tabular}

\end{minipage}%
\quad
\begin{minipage}{.5\textwidth}
\centering
\caption{Operadores binários}
\begin{tabular}{|l|c|r|}
\hline
união & 			\union \\
intersecção & 		\intersection \\
diferença & 		- \\
produto cartesiano & \cross \\
divisão & 			$\div$ \\
renomeação & 			$\rho$ \\
junção natural& 		$\bowtie$ \\
junção theta & 		$\bowtie_{\theta}$ \\
semi-junção à esquerda & 	$\leftsemijoin$ \\
semi-junção à direita & 	$\rightsemijoin$ \\ 
junção externa à esquerda & 	$\leftouterjoin$ \\
junção externa à direita & 	$\rightouterjoin$ \\
junção externa completa & 	$\fullouterjoin$ \\
anti-junção & 			? \\
\hline
\end{tabular}
    \end{minipage} 
\end{table}

Exemplo (\textit{Your \LaTeX  ~skills are likely superior to mine!}):
\begin{displaymath}
Grades \leftarrow \project_{(students.ssn, students.name, grades.grade)} \\
(\select_{students.ccn = grades.ccn \wedge grades.assignment = 1} \\
(students \cross grades)) \\
\end{displaymath}
\vskip-1em
\begin{eqnarray*}
Grades & \leftarrow & \project_{(students.ssn, students.name, grades.grade)}  \\
       &            &(\select_{(students.ssn, students.name, grades.grade)}  \\
       &            &          (students \cross grades)) 
\end{eqnarray*}