% 
%  chapter1.tex
%  
%  Created by Matilde Pós-de-Mina Pato on 2012/10/09.
%
\chapter{Introdução}
\label{cha:introduction}

\section{Linguagem e estilo}
Qualquer trabalho deverá ser escrito como um artigo, i.e. a linguagem deve ser clara, objectiva, escrita em discurso directo e com frases curtas \cite{gustavii2016write}.

A redação deve ser feita com frases curtas e objectivas, organizadas de acordo com a estrutura do trabalho, dando destaque a cada uma das partes abordadas, assim apresentadas: Introdução - Informar, em poucas palavras, o contexto em que o trabalho se insere, sintetizando a problemática estudada. Objetivo - Deve ser explicitado claramente. Métodos - Destacar os procedimentos metodológicos adoptados. Resultados - Destacar os mais relevantes para os objetivos pretendidos. Os trabalhos de natureza quantitativa devem apresentar resultados numéricos, assim como seu significado estatístico. Conclusões - Destacar as conclusões mais relevantes, os estudos adicionais recomendados e os pontos positivos e negativos que poderão influir no conhecimento. 

\paragraph{Palavras}
\begin{enumerate}
\item Use palavras precisas e específicas, simples, usuais e curtas.
\item Cuidado com a hifenização, recorra com frequência à gramática e AO. Tem disponíveis na web vários dicionários como o Dicionário Priberam da Língua Portuguesa.
\item Use apenas os adjectivos e advérbios necessários.
\item Evite repetições.
\item Não recorra a subterfúgios, como o uso de sinónimos para obter uma escrita elegante. Não os use em termos técnicos.
\item Evite assuntos laterais.
\item Evite os ecos e cacofonias, como ``Medição da orientação'' ou ``aproxima mais''
\item Evite jargões, abreviaturas sem a devida explicação ou que caíram em desuso.
\item Explique palavras científicas no texto quando as escreve a primeira vez.
\item Use o itálico, apenas em conceitos inovadores, designações expecíficas, termos científicos e noções-chave, palavras ambíguas, títulos de livros e nome de revistas científicas.
\item Não use o itálico em expressões e abreviaturas estrangeiras comuns em português, como \texttt{a priori, et al.}.
\item As abreviatura latinas devem ser usadas entre parêntesis: e.g., i.e., etc.
\item Use as aspas duplas para neologismos ou citações.
\end{enumerate}

\paragraph{Frases}
\begin{enumerate}
\item Escreva sempre no discurso directo: sujeito + verbo + complemento.
\item Prefira frases afirmativas e na voz activa, como ``Nós estudámos a (...)'' em vez de ``Foi estudado pelos investigadores (...)''.
\item Use sempre frases curtas e simples. Abuse dos pontos finais.
\item Prefira um ponto final a uma vírgula para iniciar uma nova frase. Se a informação não merece nova frase é porque não é importante e pode ser eliminada.
\item Evite as partículas de subordinação como \textit{que}, \textit{embora}. Estas palavras alongam e tornam as frases mais confusas. No máximo, use uma por frase.
\item Evite oprações intercaladas, parêntesis e travessões.
\item Quando parafrasear ou citar o trabalho de um autor, deve indicar a fonte. Caso contrário está a cometer plágio, punido pela Lei 45/85 de 17 de Setembro.
\end{enumerate}

\paragraph{Parágrafo}
\begin{enumerate}
\item Um parágrafo  deve iniciar-se com uma frase curta e que contém a informação principal. As restantes devem acompanhar o conteúdo apresentado na primeira. A última deve fazer a ligação ao parágrafo seguinte.
\item Os parágrafos devem interligar-se de forma lógica.
\end{enumerate}

\paragraph{Estrutura}

O limite de páginas para cada UC será estipulado pelo docente e pode contemplar as seguintes partes: 
\begin{enumerate}
\item[Título] curto mas não genérico.
\item[Capa] Use a capa apresentada neste documento. Complete a informação apresentada no ficheiro \texttt{datas.tex}.
\item[Resumo] Faça um resumo dos conteúdos do trabalho e apresente as conclusões básicas, \texttt{resumo.tex} e \texttt{abstract.tex} (se necessário, c.c. comente a linha 155 de \texttt{template.tex}).
\item[Índice] Indique as páginas dos títulos e subtítulos, figuras ou tabelas. O código que suporta algum parágrafo deve constar no índice respectivo, de Listagens.
\item[Introdução] Contextualize o tema e indique o objectivo de estudo.
\item[Desenvolvimento] Descreva as definições, modelos e teorias suportados por referências bibliográficas. 
\item[Conclusão] Sintetize os aspectos relevantes,
\item[Bibliografia] Escreva todas as referências indicadas no texto.
\item[Anexos] Use os anexos para colocar outras informações que considere oportunas, mas não relevantes o suficiente para colocar no corpo do documento. 
\end{enumerate}

\paragraph{Bibliografia}
As referências são listadas pela ordem alfabética do apelido dos autores e depois por ordem cronológica quando o nome se repetir, \textit{package: plainnat}. 

\begin{enumerate}
\item[Livros] Deve constar o nome original de um livro escrito em língua estrangeira. Pede-se que seja inserido as páginas consultadas. 
\item[websites] Os sítios da internet consultados também devem constar nas referências. Pede-se que seja introduzido o dia de consulta do mesmo.
\item[Artigos] Deve citar aqueles que se encontram indexados e submetidos a revisão independente. No instituto Thomson Reuters são fornecidas listas de toda a bibliografia que obedece a esse grau de exigência.
\item[Trabalhos] Se o trabalho citado não tiver data, coloque o nome do autor seguido da indicação ``sem data''. Se a citação for relativa a uma comunicação pessoal, então faça-o do modo seguinte: M. Mjhdsh (comunicação pessoal, 13 de Março 2017). Noutros trabalhos não publicados, deve constar a seguinte informação: ``Dissertação (ou, Relatório) de Mestrado (ou Doutoramento, ou Final de Curso) não publicada(o). 
\end{enumerate}

Por uma questão de simplificação, pode recorrer ao  Google Académico\footnote{scolar.google.pt/} e retirar a informação que consta no sítio \texttt{Citar} no formato BibTeX e inserir no ficheiro \texttt{bibliography.bib}. Por exemplo:
\begin{verbatim}
@book{gustavii2016write,
  title={How to write and illustrate a scientific paper},
  author={Gustavii, Bj{\"o}rn},
  year={2016},
  publisher={Cambridge University Press}
}
\end{verbatim}