\chapter{Tutorial de \texttt{LaTeX} com exemplos}
\label{cha:a_short_latex_tutorial_with_examples}

Este capítulo irá mostrar como pode inserir tabelas, figuras e/ou listagens. Consi\-de\-re-se que listagens não são todo o código mas partes do código. Estas irão suportar algumas frases dos parágrafos escritos. Não se esqueça, que todas as tabelas, figuras e listagens devem ser referenciadas no corpo do texto.

\section{Tabelas} % (fold)
\label{sec:inserting_tables}

Some notes are important to followed, such as present in \tablename~  \ref{tab:results}: 
\begin{inparaenum}[i)]
	\item Not defined vertical lines;
	\item The legend must be on top;
	\item Use \verb!\toprule!, \verb!\midrule! and \verb!\bottomrule! to draw horizontal lines.
\end{inparaenum}
 
\begin{table}[ht]
	\caption{Table's rules.}\vskip0.5em
	\label{tab:results}
\centering
\begin{tabular}{llr}
\toprule
\multicolumn{2}{c}{Item} \\
\cmidrule(r){1-2}
Animal & Description & Price (\$) 
\\
\midrule
Gnat  & per gram & 13.65 \\
      & each     &  0.01 \\
Gnu   & stuffed  & 92.50 \\
Emu   & stuffed  & 33.33 \\
\bottomrule
\end{tabular}
\end{table}


\section{Importar Figuras} % (fold)
\label{sec:importing_images}

\subsection{Inserir figuras lado-a-lado com texto} % (fold)
 \label{ssec:inserting_images_wrapped_with_text}
 
 You should only use this feature is \emph{really} necessary. This means, you have a very small image, that will look lonely just with text above and below.

 \begin{wrapfigure}{l}{3cm}
   \centering
     \includegraphics[width=2cm]{evolution_steps-vectorial}
   \caption{Vectorial image}\
 \end{wrapfigure}	
 
 \noindent\verb!\usepackage{wrapfig}!\\
 This then gives you access to:\\
 \verb!\begin{wrapfigure}[lineheight]{alignment}{width}!
 
 Alignment can normally be either 'l' for left, or 'r' for right. Lowercase 'l' or 'r' forces the figure to start precisely where specified (and may cause it to run over page breaks), while capital 'L' or 'R' allows the figure to float. If you defined your document as twosided, the alignment can also be 'i' for inside or 'o' for outside, as well as 'I' or 'O'. The width is obviously the width of the figure. \\[.5em]

 
 
\subsection{Outras imagens} % (fold)
\label{ssec:floats_figures_and_captions}

 
There are two different ways to place two figures/tables side-by-side.  More complicated figures with multiple images. You can do this using subfigure environments inside a figure environment. Subfigure will alphabetically number your subfigures and you have access to the complete reference as usual through \verb!\ref{fig:figurelabel}!, \figurename~\ref{fig:figura-completa}, or \figurename~\ref{fig:ra-raster} using \verb!\ref{fig:subfigurelabel}!.

\begin{figure}[htbp]
	\centering
    \begin{subfigure}[b]{0.25\textwidth}
    	\centering
		\includegraphics[width=\textwidth]{ra-vectorial}
		\caption{}
		\label{fig:ra-vectorial}
     \end{subfigure}	
\qquad\qquad
 	\begin{subfigure}[b]{0.25\textwidth}
    	\centering
		\includegraphics[width=\textwidth]{ra-raster}
		\caption{}
		\label{fig:ra-raster}
	\end{subfigure}		
  \caption{Subfigure example with vectorial and no-vectorial images}
  \label{fig:figura-completa}
\end{figure}

\section{Listagens} % (fold)
\label{sec:listings}

Using the package listings you can add non-formatted text as you would do with \verb!\begin{verbatim}! but its main aim is to include the source code of any programming language within your document, or inline with 
\verb!\codejava{int i;}.!
If you wish to include pseudocode or algorithms see \url{http://en.wikibooks.org/wiki/LaTeX/Algorithms\_and\_Pseudocode}, as \lstlistingname ~\ref{lst:Calculadora}.
\vskip1em

%  \lstfromfile{language}{linerange}{caption}{label}{othersinsideoflstinoputlisting}{path}

\lstfromfile{java}{7,11-12,15-17}{Opera\c{c}\~{o}es elementares de uma Calculadora}{Calculadora}{showlines=true,morekeywords={begin,System,out,print},numbers=left, firstnumber=1}{Calculadora.java}


\centering{\textbf{Written by Matilde Pós-de-Mina Pato\footnote{O autor escreve segundo o antigo AO}, \\2020 September -  version 2.2}}