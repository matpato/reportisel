% 
%  chapter2.tex
%  ReportISEL
%  
%  Created by Matilde Pós-de-Mina Pato on 2012/10/09.
%
\chapter{Manual do utilizador}
\label{cha:users_manual}

% ================
% = Introduction =
% ================
\section{Introdução} % (fold)
\label{sec:introduction}
Este documento reutiliza algumas partes do \textit{template} criado para a escrita da Dissertação/Trabalho de Projecto de Mestrado de MEIC. Por ser meramente informativo, o resto do documento encontra-se escrito na lingua inglesa.

% section introduction (end)

% ====================
% = Folder Structure =
% ====================
\section{Folder Structure} % (fold)
\label{sec:folder_structure}

The template file for writing dissertations in  \texttt{LaTeX} is organized into a main directory, a set of files and sub-directories. \textbf{reportisel}  is the main directory and includes:
	\begin{description}
		\item[Logo] Directory with Faculty logos;
		\item[Chapters] Directory where to put user files (text and figures);
		\begin{description}
		\item[scripts] Directory with useful bash scripts, e.g., for cleaning all temporary files;
		\item[img] Directory with all images to use in the chapters, e.g. \texttt{ra-raster.png};
		\end{description}
		\item[alpha-pt.bst] A file with bibliography names in portuguese, e.g., 'Relatório Técnico'. This file is used automatically if Portuguese is selected as the main language (see below);
		\item[defaults.tex] A file with the main default values for the package (institution name, faculty's logo, degree name and similars - {\color{red}{TO CHANGE BY THE USER}});
		\item[personaldataofreport.tex] A file with the main default values for the package (identification of report as well as the author and juries - {\color{red}{TO CHANGE BY THE USER}});
		\item[template.tex] The main file. You should run  \texttt{LaTeX} in this one. Please refrain from changing the file content outside of the well defined area;
		\item[bibliography.bib] The bib file. An easy way to find to import citation into \texttt{bibtex} is select option \texttt{Show links to import citation into \texttt{BibTex}} in \href{http://scholar.google.pt/scholar_settings?hl=en&as_sdt=0,5}{\texttt{Scholar google settings}}.
		\item[reportisel.cls] The  \texttt{LaTeX} class file for the thesis{} style. Currently, some of the defaults are stored here instead of \verb!defaults.tex!. This file should not be changed, unless you're ready to play with fire! :) - {\color{red}{DO NOT CHANGE}});
	\end{description}

Again, we would like to recall that all the user \texttt{LaTeX} files should be stored in the \verb!reportisel! directory, and all the images in \verb!reportisel/Chapters/img! directory.\todo[inline]{Yet another note!}
% section folder_structure (end)

% ===================
% = Package options =
% ===================
\section{Package Options} % (fold)
\label{sec:package_options}

The thesis{} style includes the following options, that must be included in the options list in the \verb!\documentclass[options]{reportisel}! line at the top of the \texttt{template.tex} file.

The list below aggregates related options in a single item. For each list, the default value is prefixed with a *.

\subsection{Language Related Options} % (fold)
\label{sub:language_related_options}

You must choose the main language for the document. The available options are:

\begin{enumerate}
	\item \textbf{*pt} --- The text is written in Portuguese (with a small abstract in English).
	\item \textbf{en} --- The text is written in English (with a small abstract in Portuguese).
\end{enumerate}

The language option affects:
\begin{itemize}
	\item \textbf{The order of the summaries.} At first the abstract in the main language and then in the foreign language. This means that if your main language for the document in english, you will see first the abstract (in english) and then the 'resumo' (in portuguese). If you switch the main language for the document, it will also automatically switch the order of the summaries.
	\item \textbf{The names for document sectioning.} E.g., 'Chapter' vs.\ 'Capítulo', 'Table of Contents' vs.\ 'Índice', 'Figure' vs.\ 'Figura', etc.
	\item \textbf{The type of documents in the bibliography.} E.g., 'Technical Report' vs.\ 'Relatório Técnico'.
\end{itemize} 

No mater which language you chose, you will always have the appropriate hyphenation rules according to the language at that point. You always get portuguese hyphenation rules in the 'Resumo', english hyphenation rules in the 'Abstract', and then the main language hyphenation rules for the rest of the document. If you need to force hyphenation write inside of \verb!\hyphenation{}! the hyphenated word, e.g. \verb!\hyphenation{op-ti-cal net-works}!.
% section package_options (end)

\subsection{Class of Text} % (fold)
\label{sub:class_of_text}

You must choose the class of text for the document. The available options are:

\begin{enumerate}
	\item \textbf{rpt} --- BSc report.
	\item \textbf{preprpt} --- Preparation of Bsc report. This is a preliminary report graduate students at ISEL/IPL must prepare to conclude the first semester.
\end{enumerate}
%% subsection class_of_text (end)
%
%% ============
%% = Printing =
%% ============
\subsection{Printing} % (fold)
\label{sub:printing}

You must choose how your document will be printed. The available options are:
\begin{enumerate}
	\item \textbf{oneside} --- Single side page printing.
	\item \textbf{*twoside} --- Double sided page printing.
\end{enumerate}
% subsection printing (end)

% =============
% = Font Size =
% =============
\subsection{Font Size} % (fold)
\label{ssec:font_size}

You must select the encoding for your text. The available options are:
\begin{enumerate}
	\item \textbf{11pt} --- Eleven (11) points font size.
	\item \textbf{*12pt} --- Twelve (12) points font size. You should really stick to 12pt\ldots
\end{enumerate}
% subsection font_size (end)

% =================
% = Text encoding =
% =================
\subsection{Text Encoding} % (fold)
\label{ssec:text_encoding}

You must choose the font size for your document. The available options are:
\begin{enumerate}
	\item \textbf{latin1} --- Use Latin-1 (\href{http://en.wikipedia.org/wiki/ISO/IEC_8859-1}{ISO 8859-1}) encoding.  Most probably you should use this option if you use Windows;
	\item \textbf{utf8} --- Use \href{http://en.wikipedia.org/wiki/UTF-8}{UTF8} encoding.    Most probably you should use this option if you are not using Windows.
\end{enumerate}
% subsection font_size (end)

% ============
% = Examples =
% ============
\subsection{Examples} % (fold)
\label{ssec:examples}

Let's have a look at a couple of examples:

\begin{itemize}
	\item Preparation of report document, in portuguese, with 12pt size, to be printed twoside sided  and to be read on screen
	\begin{lstlisting}[language=myLatex]
	\documentclass[
		preprpt,	% (*)rpt, preprpt - Technical Report or PrepTechnical Report
		pt,	% (*)pt, en - languages 
		twoside,	% (*)twoside, oneside - single or double sided printing
		12pt,	% (*)12pt, 11pt, 10pt - use font size
		a4paper,	% Paper size/format
		utf8,		% (*)utf8, latin1 - Text encoding: Linux, Mac or Windows
		onscreen,	% (*)onscreen, onpaper - Page layout: screen versus paper print
		hyperref = true,  % (*)true, false - Hyperlinks in citations
		listof = totoc	% Print all entries in table of contents
	]{reportisel} 
	\end{lstlisting}
	\item Report, in english, with 12pt size and to be printed one sided (I wonder why one would do this!), in paper
	\begin{lstlisting}[language=myLatex]
	\documentclass[
		rpt,	% (*)rpt, preprpt - Technical Report or PrepTechnical Report
		pt,	% (*)pt, en - languages 
		oneside,	% (*)twoside, oneside - single or double sided printing
		12pt,	% (*)12pt, 11pt, 10pt - use font size
		a4paper,	% Paper size/format
		utf8,		% (*)utf8, latin1 - Text encoding: Linux, Mac or Windows
		onpaper,	% (*)onscreen, onpaper - Page layout: screen versus paper print
		hyperref = true,  % (*)true, false - Hyperlinks in citations
		listof = totoc	% Print all entries in table of contents
	]{reportisel} 
	\end{lstlisting}
\end{itemize}
% subsection examples (end)

\section{How to Write Using \texttt{LaTeX}} % (fold)
\label{sec:how_to_write_using_latex}

Please have a look at Chapter~\ref{cha:a_short_latex_tutorial_with_examples}, where you may find many examples of \href{http://tobi.oetiker.ch/lshort/lshort.pdf}{\texttt{LaTeX}} constructs, such as Sectioning, inserting Figures and Tables, writing Equations, Theorems and algorithms, exhibit code listings, etc.

%% section how_to_write_using_latex (end)
