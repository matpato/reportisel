\abstractPT  % Do NOT modify this line

O objectivo deste documento é uniformizar a apresentação dos trabalhos escolares de licenciatura, trabalhos de investigação científicos e dissertações de mestrado.

Independentemente da língua em que está escrita o documento, é necessário um resumo na língua do texto principal e um resumo noutra língua.  Assume-se que as duas línguas em questão serão sempre o Português e o Inglês.

O \emph{template} colocará automaticamente em primeiro lugar o resumo na língua do texto principal e depois o resumo na outra língua.  Por exemplo, se  está escrita em Português, primeiro aparecerá o resumo em Português, depois em Inglês, seguido do texto principal em Português. 

Resumo é a versão precisa, sintética e selectiva do texto do documento, destacando os elementos de maior importância. O resumo possibilita a maior divulgação do trabalho e a sua indexação em bases de dados.

A redação deve ser feita com frases curtas e objectivas, organizadas de acordo com a estrutura do trabalho, dando destaque a cada uma das partes abordadas, assim apresentadas: Introdução - Informar, em poucas palavras, o contexto em que o trabalho se insere, sintetizando a problemática estudada. Objetivo - Deve ser explicitado claramente. Métodos - Destacar os procedimentos metodológicos adoptados. Resultados - Destacar os mais relevantes para os objetivos pretendidos. Os trabalhos de natureza quantitativa devem apresentar resultados numéricos, assim como seu significado estatístico. Conclusões - Destacar as conclusões mais relevantes, os estudos adicionais recomendados e os pontos positivos e negativos que poderão influir no conhecimento. 

O resumo não deve conter citações bibliográficas, tabelas, quadros, esquemas. Dar preferência ao uso dos verbos na 3ª pessoa do singular. Tempo e verbo não devem dissociar-se dentro do resumo. Deve evitar o uso de abreviaturas e siglas - quando absolutamente necessário, citá-las entre parênteses e precedidas da explicação de seu significado, na primeira vez em que aparecem. 

E, deve-se evitar o uso de expressões como "O presente trabalho trata ...", "Nesta tese são discutidos....", "O documento conclui que....", "aparentemente é...." etc. 

Existe um limite de palavras, 300 palavras é o limite.

