\abstractPT  % Do NOT modify this line

O objectivo deste documento é uniformizar a apresentação dos trabalhos escolares de licenciatura, trabalhos de investigação científicos e dissertações de mestrado.

Independentemente da língua em que está escrita o documento, é necessário um resumo na língua do texto principal e um resumo noutra língua.  Assume-se que as duas línguas em questão serão sempre o Português e o Inglês.

O \emph{template} colocará automaticamente em primeiro lugar o resumo na língua do texto principal e depois o resumo na outra língua.  Por exemplo, se  está escrita em Português, primeiro aparecerá o resumo em Português, depois em Inglês, seguido do texto principal em Português. 

Resumo é a versão precisa, sintética e selectiva do texto do documento, destacando os elementos de maior importância. O resumo possibilita a maior divulgação do trabalho e a sua indexação em bases de dados.

O resumo não deve conter citações bibliográficas, tabelas, quadros, esquemas. Dar preferência ao uso dos verbos na 3ª pessoa do singular. Tempo e verbo não devem dissociar-se dentro do resumo. Deve evitar o uso de abreviaturas e siglas - quando absolutamente necessário, citá-las entre parênteses e precedidas da explicação de seu significado, na primeira vez em que aparecem. 

E, deve-se evitar o uso de expressões como ``O presente trabalho trata ...'', ``Nesta tese são discutidos....'', ``O documento conclui que....'', ``aparentemente é....'' etc. 

Existe um limite de palavras, 300 palavras é o limite.

\begin{center}
	\textbf{\large this package and template are not official for ISEL/IPL}.
\end{center}